\chapter{User's guide} 
\label{chap:user} 

This section is generated automatically from \texttt{README.rst}, the same file that produces the \href{github.com/malaria-atlas-project/generic-mbg}{GitHub documentation}. 


\section{Detailed usage instructions%
}

If you want to use the shell commands, this section is for you.




\subsection{\texttt{mbg-infer}%
}
%
\begin{quote}{\ttfamily \raggedright \noindent
mbg-infer~module~database-file~input~{[}options{]}
}
\end{quote}

Produces the requested database file. Also produces plots of the dynamic traces of all
scalar parameters as PDF's, and saves them in the folder \texttt{name-plots}, where \texttt{name}
is the name of the database file. You will need to inspect these plots to determine how
many 'burnin' iterations should be discarded when making maps.

If you determine that more MCMC samples are needed, simply run mbg-infer with the same
database file argument to pick up where you left off and keep sampling.




\subsubsection{Required arguments%
}
\newcounter{listcnt0}
\begin{list}{\arabic{listcnt0}.}
{
\usecounter{listcnt0}
\setlength{\rightmargin}{\leftmargin}
}

\item The name of the module containing the model specification.

\item The name of the database file to be produced. If you do not want it to go in the current
directory, specify a path, eg \texttt{/home/anand/traces/run-01-04-2009}. If the database file
already exists, you will be prompted about whether you want to continue sampling into it
or remove it.

\item The name of a csv file containing the input data. If it is a different directory, specify
the path to it, eg \texttt{/home/anand/data/query-01-04-2009.csv}. This csv file must have the
following columns:
%
\begin{itemize}

\item \texttt{lon}, \texttt{lat} : The coordinates of the observation in decimal degrees

\item \texttt{t} : Time in decimal years. This is only required for spatiotemporal models.

\end{itemize}

All other columns are interpreted as covariates, eg \texttt{ndvi} etc., UNLESS the module
implements the \texttt{non\_cov\_columns} attribute. For example, MBGWorld expects
lo\_age, up\_age columns, pos and neg columns, but does not interpret them as covariates.
\end{list}




\subsubsection{Options%
}
%
\begin{itemize}

\item \texttt{-t} or \texttt{-{}-thin} : If thin is 10, every 10th MCMC iteration will be stored in the
database. Small values are good but slow. 1 is best.

\item \texttt{-i} or \texttt{-{}-iter} : The number of MCMC iterations to perform. Large values are good
but slow.

\end{itemize}

% * ``-n`` or ``-ncpus`` : The maximum number of CPU cores to make available to the MCMC

% algorithm. Should be less than or equal to the number of cores in your computer. The

% All the cores you make available may not be utilized. Use top or the Activity Monitor

% to monitor your actual CPU usage. Large values are good but tie up more of your computer.




\subsection{\texttt{mbg-describe-tracefile}%
}
%
\begin{quote}{\ttfamily \raggedright \noindent
mbg-describe-tracefile~path
}
\end{quote}

If path is a database file, inspects the database file. Prints out the version of the
generic package, the module that produced the file and the date the run was started.
Writes the input data to csv with filename \texttt{database-file-input-csv}, substituting
the actual filename.

If the path is a directory, walks the filesystem starting from the directory, inspecting
every database file it finds. Does not produce any csvs.




\subsubsection{Required arguments%
}
\setcounter{listcnt0}{0}
\begin{list}{\arabic{listcnt0}.}
{
\usecounter{listcnt0}
\setlength{\rightmargin}{\leftmargin}
}

\item The name of the database file or path to be inspected.
\end{list}




\subsection{\texttt{mbg-covariate-traces}%
}
%
\begin{quote}{\ttfamily \raggedright \noindent
mbg-covariate-traces~module~database-file~{[}options{]}
}
\end{quote}

Postprocesses the given database file to produce MCMC traces for the covariate
coefficients. Produces a directory called database-file-covariate-traces, and populates
it with pdf images of the covariate coefficient traces and




\subsubsection{Required arguments%
}
\setcounter{listcnt0}{0}
\begin{list}{\arabic{listcnt0}.}
{
\usecounter{listcnt0}
\setlength{\rightmargin}{\leftmargin}
}

\item The name of the module containing the model specification.

\item The name of the database file containing the MCMC trace.
\end{list}




\subsubsection{Options%
}
%
\begin{itemize}

\item \texttt{-t} or \texttt{-{}-thin} : If thin is 10, samples of the covariate coefficients will be
produced for every 10th MCMC sample. Defaults to 1, meaning no thinning.

\item \texttt{-b} or \texttt{-{}-burn} : Samples of the covariate coefficients will begin after this
many 'burnin' iterations are discarded. Defaults to 0, meaning no burnin.

\end{itemize}




\subsection{\texttt{mbg-decluster}%
}
%
\begin{quote}{\ttfamily \raggedright \noindent
mbg-decluster~input~prop~{[}options{]}
}
\end{quote}

A wrapper for the R function getdeclusteredsample that results in two new tables with
suffix HOLDOUT and THINNED outut to same directory as tablepath




\subsubsection{Required arguments%
}
\setcounter{listcnt0}{0}
\begin{list}{\arabic{listcnt0}.}
{
\usecounter{listcnt0}
\setlength{\rightmargin}{\leftmargin}
}

\item (string) path to input table. must include columns 'lon' and 'lat'. If
also 't' will treat as space-time. If only filename given (no path) assumes file
in current working directory.

\item (float) what proportion of the full data set will be used for hold-out set.
\end{list}




\subsubsection{Options%
}
%
\begin{itemize}

\item \texttt{-m} or \texttt{-{}-minsample} : (int) optional minimum sample size (supercedes prop.
if larger)

\item \texttt{-d} or \texttt{-{}-decluster} : (logical) do we want to draw spatially declustered
sample (default) or just simple random.

\item \texttt{-p} or \texttt{-{}-makeplot} : (logical) do we want to export a pdf map showing
location of data and selected points. This is exported to same directory as
tablepathoptional minimum sample size (supercedes prop if larger).

\end{itemize}




\subsection{\texttt{mbg-map}%
}
%
\begin{quote}{\ttfamily \raggedright \noindent
mbg-map~module~database-file~burn~mask~{[}options{]}
}
\end{quote}

Produces a folder called \texttt{name-maps} where \texttt{name} is the name of the database file.
Puts the requested maps in the folder in format matching the mask. Also produces PDF
images of all the requested maps for quick viewing.




\subsubsection{Required arguments%
}
\setcounter{listcnt0}{0}
\begin{list}{\arabic{listcnt0}.}
{
\usecounter{listcnt0}
\setlength{\rightmargin}{\leftmargin}
}

\item The name of the module containing the model specification.

\item The name of the database file (produced by mbg-infer) to be used to generate the
maps. If you do not want it to go in the current directory, specify a path.

\item The number of burnin iterations to discard from the trace before making the maps.
You will need to figure this out by inspecting the traces produced by \texttt{mbg-infer}.

\item The name of a raster, without extension. The maps will be produced in raster files
in the same format, on identical grids, with identical missing pixels. If the file
is in a different directory, specify the path to it.
\end{list}




\subsubsection{Options%
}
%
\begin{itemize}

\item \texttt{-n} or \texttt{-{}-n-bins} : The number of bins to use in the histogram from which quantiles
are computed. Large values are good, but use up more system memory. Decrease this if you
see memory errors.

\item \texttt{-b} or \texttt{-{}-bufsize} : The number of buffer pixels to render around the edges of the
continents. Set to zero unless the \texttt{raster-thin} option is greater than 1. The buffer
will not be very good. In general, if you want a buffer you're better off making your
own in ArcView rather than using this option.

\item \texttt{-q} or \texttt{-{}-quantiles} : A string containing the quantiles you want. For example,
\texttt{'0.25 0.5 0.75'} would map the lower and upper quartiles and the medial. Default is
\texttt{'0.05 0.25 0.5 0.75 0.95'}.

\item \texttt{-t} or \texttt{-{}-thin} : The factor by which to thin the MCMC trace stored in the database.
If you use \texttt{-t 10}, only every 10th stored MCMC iteration will be used to produce the maps.
Small values are good but slow. 1 is best.

\item \texttt{-i} or \texttt{-{}-iter} : The total number of predictive samples to use in generating the maps.
Large values are good but slow. Defaults to 20000.

\item \texttt{-p} or \texttt{-{}-raster-path} : The path to the files containing the covariate rasters. These
files' headers must match those of the input raster, and their missing pixels must match
those of the input raster also. There must be a file corresponding to every covariate column
in input 3 of mbg-infer. For example, if you used \texttt{rain} and \texttt{ndvi} as your column headers,
files \texttt{rain.asc} and \texttt{ndvi.flt} and \texttt{temp.hdf5} should be present in the raster path.
Defaults to the current working directory.

\item \texttt{-y} or \texttt{-{}-year} : If your model is spatiotemporal, you must provide the decimal year at
which you want your map produced. For example, Jan 1 2008 would be \texttt{-y 2008}.

\end{itemize}




\subsection{\texttt{mbg-3dmap}%
}
%
\begin{quote}{\ttfamily \raggedright \noindent
mbg-3dmap~module~database-file~burn~mask~{[}options{]}
}
\end{quote}

Produces a folder called \texttt{name-3dmaps} where \texttt{name} is the name of the database file.
Puts a HDF5 file- containing the probability density field of the output of each function
in the specializing module's \texttt{map\_postproc} list in the folder. This data can be examined
interactively using MayaVI. File \texttt{display\_3dmap.py}, included with the package, provides
a template for scene generation.




\subsubsection{Required arguments%
}
\setcounter{listcnt0}{0}
\begin{list}{\arabic{listcnt0}.}
{
\usecounter{listcnt0}
\setlength{\rightmargin}{\leftmargin}
}

\item The name of the module containing the model specification.

\item The name of the database file (produced by mbg-infer) to be used to generate the
maps. If you do not want it to go in the current directory, specify a path.

\item The number of burnin iterations to discard from the trace before making the maps.
You will need to figure this out by inspecting the traces produced by \texttt{mbg-infer}.

\item The name of a raster, without extension. The maps will be produced in raster files
in the same format, on identical grids, with identical missing pixels. If the file
is in a different directory, specify the path to it.
\end{list}




\subsubsection{Options%
}
%
\begin{itemize}

\item \texttt{-n} or \texttt{-{}-n-bins} : The number of bins to use in the histogram from which quantiles
are computed. Large values are good, but use up more system memory. Decrease this if you
see memory errors.

\item \texttt{-b} or \texttt{-{}-bufsize} : The number of buffer pixels to render around the edges of the
continents. Set to zero unless the \texttt{raster-thin} option is greater than 1. The buffer
will not be very good. In general, if you want a buffer you're better off making your
own in ArcView rather than using this option.

\item \texttt{-q} or \texttt{-{}-quantiles} : A string containing the quantiles you want. For example,
\texttt{'0.25 0.5 0.75'} would map the lower and upper quartiles and the medial. Default is
\texttt{'0.05 0.25 0.5 0.75 0.95'}.

\item \texttt{-t} or \texttt{-{}-thin} : The factor by which to thin the MCMC trace stored in the database.
If you use \texttt{-t 10}, only every 10th stored MCMC iteration will be used to produce the maps.
Small values are good but slow. 1 is best.

\item \texttt{-r} or \texttt{-{}-raster-thin}: The 3d data cube takes up much more disk space and memory than
the scalar maps. You might need to degrade the input raster to lower resolution. A value of
10 means that the 3d maps will have 1/10 the spatial resolution of the input raster.

\item \texttt{-i} or \texttt{-{}-iter} : The total number of predictive samples to use in generating the maps.
Large values are good but slow. Defaults to 20000.

\item \texttt{-p} or \texttt{-{}-raster-path} : The path to the files containing the covariate rasters. These
files' headers must match those of the input raster, and their missing pixels must match
those of the input raster also. There must be a file corresponding to every covariate column
in input 3 of mbg-infer. For example, if you used \texttt{rain} and \texttt{ndvi} as your column headers,
files \texttt{rain.asc} and \texttt{ndvi.flt} and \texttt{temp.hdf5} should be present in the raster path.
Defaults to the current working directory.

\item \texttt{-y} or \texttt{-{}-year} : If your model is spatiotemporal, you must provide the decimal year at
which you want your map produced. For example, Jan 1 2008 would be \texttt{-y 2008}.

\end{itemize}




\subsection{\texttt{mbg-validate}%
}
%
\begin{quote}{\ttfamily \raggedright \noindent
mbg-validate~module~database-file~burn~pred-pts~{[}options{]}
}
\end{quote}

mbg-validate produces a folder called \texttt{name-validation}, \texttt{name} being the name of the database file.
It populates this folder with two csv files called \texttt{p-samps} and \texttt{n-samps} containing posterior
predictive samples of the probability of positivity and the number of individuals positive at each
prediction location.

It also writes three of the four MBG world validation panels into the folder as PDF's.




\subsubsection{Required arguments%
}
\setcounter{listcnt0}{0}
\begin{list}{\arabic{listcnt0}.}
{
\usecounter{listcnt0}
\setlength{\rightmargin}{\leftmargin}
}

\item The name of the module containing the model specification.

\item The name of the database file (produced by mbg-infer) to be used to generate the
maps. If you do not want it to go in the current directory, specify a path.

\item The number of burnin iterations to discard from the trace before making the maps.
You will need to figure this out by inspecting the traces produced by \texttt{mbg-infer}.

\item A csv file containing the 'holdout' dataset. It should be in exactly the same format
as the third required input to \texttt{mbg-infer}.
\end{list}




\subsubsection{Options%
}
%
\begin{itemize}

\item \texttt{-t} or \texttt{-{}-thin} : The factor by which to thin the MCMC trace stored in the database.
Small values are good but slow. 1 is best.

\item \texttt{-i} or \texttt{-{}-iter} : The total number of predictive samples you want to generate. Large
values are good but slow. Defaults to 20000.

\end{itemize}




\subsection{\texttt{mbg-scalar-priors}%
}
%
\begin{quote}{\ttfamily \raggedright \noindent
mbg-scalar-priors~module~{[}options{]}
}
\end{quote}




\subsubsection{Required arguments%
}
\setcounter{listcnt0}{0}
\begin{list}{\arabic{listcnt0}.}
{
\usecounter{listcnt0}
\setlength{\rightmargin}{\leftmargin}
}

\item The name of the module containing the model specification.
\end{list}




\subsubsection{Options%
}
%
\begin{itemize}

\item \texttt{-i} or \texttt{-{}-iter} : The total number of predictive samples you want to generate. Large
values are good but slow. Defaults to 20000.

\end{itemize}




\subsection{\texttt{mbg-realize-prior}%
}
%
\begin{quote}{\ttfamily \raggedright \noindent
mbg-realize-prior~module~ascii0.asc~ascii1.asc~...~{[}options{]}
}
\end{quote}

mbg-realize-prior produces a number of prior realizations of the target surface (eg parasite
rate, gene frequency, etc). on several different asciis. Joint or 'conditional' simulations
of surfaces are very expensive, so you can only afford to evaluate them on a few thousand
pixels.

The multiple asciis are meant to be at multiple resolutions: you can make a coarse one over
your entire area of interest, a medium-resolution one on a zoomed-in subset, and a few fine
ones over small areas scattered around. That way you can see the large- and small-scale
properties of the surface allowed by your prior without having to render the entire surface
at full resolution.

Outputs a number of surfaces, evaluated onto the masks indicated by the input asciis. Each set
of realizations is coherent across the input asciis; that is, the 'same' surface is evaluated
on each ascii. That means you can meaningfully overlay the output asciis at different
resolutions.

NOTE: All the parameters of the model will be drawn from the prior before generating each
realization. If you want to fix a variable, you must set its \texttt{observed} flag.




\subsubsection{Required arguments%
}
\setcounter{listcnt0}{0}
\begin{list}{\arabic{listcnt0}.}
{
\usecounter{listcnt0}
\setlength{\rightmargin}{\leftmargin}
}

\item The name of the module containing the model specification.

\item Several ascii files. Realizations will be evaluated on the union of the unmasked regions
of these files.
\end{list}




\subsubsection{Options%
}
%
\begin{itemize}

\item \texttt{-n} or \texttt{-{}-n-realizations} : The number of realizations to generate. Defaults to 5.

\item \texttt{-m} or \texttt{-{}-mean} : The value of the global mean to use. Defaults to 0.

\item \texttt{-y} or \texttt{-year} : If your model is spatiotemporal, you must provide the decimal year at
which you want your realizations produced. For example, Jan 1 2008 would be \texttt{-y 2008}.

\end{itemize}
